\documentclass[11pt]{article}
\usepackage{amsmath}
\usepackage{graphicx}
\usepackage{hyperref}
\usepackage{mathtools}
\usepackage[latin1]{inputenc}
% \usepackage{exercise}
\usepackage{cancel}
\usepackage[margin=1.25cm]{geometry}
\usepackage{float}
\usepackage{multirow}
% \usepackage{wrapfig}
% \usepackage{amssymb}
% \usepackage{subfigure}
\setlength\parindent{0pt}

\title{Computational Neuroscience Project - Group 04}
\author{Giuseppe Leo, Francesco Negri}
\date{A.Y. 2022-2023}

\begin{document}
\maketitle

\section*{Introduction}
The Leaky Integrate-and-Fire (LIF) and Integrated-and-Fire-or-Burst (IFB) models
are abstract neuronal models, aiming at reproducing the electrical physiological
behaviour of neurons.\\
Also, every time the membrane potential \(V_{m}\) overcomes the threshold voltage
\(V_{th}\) a spike is fired and the membrane potential goes back to \(V_{reset}\).\\
The LIF model is defined as followed:
\begin{align*}
    C_{m}\frac{dV_{m}}{dt} & =-I_{leak}+I_{stim}                           \\
                           & =-g_{leak}\bigl(V_{m}-E_{leak}\bigr)+I_{stim}
\end{align*}
On the other hand, the IFB neuron is modelled by these two differential equations:
\begin{align*}
    C_{m}\frac{dV_{m}}{dt}       & =-I_{leak}-I_{T}+I_{stim}                                                                                \\
                                 & =-g_{leak}\bigl(V_{m}-E_{leak}\bigr)-g_{T}h\bigl(V_{m}-E_{T}\bigr)\Theta\bigl(V_{m}-V_{h}\bigr)+I_{stim} \\
    \tau_{h}(V_{m})\frac{dh}{dt} & =-h+h_{\infty}(V_{m})
\end{align*}
where \(h_{\infty}(V_{m})=\Theta\bigl(V_{h}-V_{m}\bigr)\),
\(\tau_{h}(V_{m})=\tau_{h}^{-}\Theta\bigl(V_{m}-V_{h}\bigr)+\tau_{h}^{+}\Theta\bigl(V_{h}-V_{m}\bigr)\) and
\(\Theta\bigl(\,\cdot\,\bigr)\) is the Heaviside step function.\\
The two models are parametrized as shown in the table below:
\begin{table}[!h]
    \begin{center}
        \begin{tabular}{ |c||c|c|  }
            \hline
            \multicolumn{3}{|c|}{Models Parameters}                      \\
            \hline
            {}               & LIF neuron  & IFB neuron                  \\
            \hline
            \(S\)            & -           & \(3\cdot{10^{-4}}\;cm^{2}\) \\
            \(C_{m}\)        & \(100\;pF\) & \(2\;\mu{F}/cm^{2}\)        \\
            \(V_{0}\)        & \(-70\;mV\) & \(-58\;mV,\;\;-77\;mV\)     \\
            \(V_{reset}\)    & \(-70\;mV\) & \(-60\;mV\)                 \\
            \(V_{th}\)       & \(-50\;mV\) & \(-50\;mV\)                 \\
            \(V_{h}\)        & -           & \(-70\;mV\)                 \\
            \(V_{spike}\)    & \(20\;mV\)  & \(20\;mV\)                  \\
            \(g_{leak}\)     & \(10\;nS\)  & \(0.035\;mS/cm^{2}\)        \\
            \(E_{leak}\)     & \(-70\;mV\) & \(-75\;mV\)                 \\
            \(g_{T}\)        & -           & \(0.8\;mS/cm^{2}\)          \\
            \(E_{T}\)        & -           & \(120\;mV\)                 \\
            \(T_{ref}\)      & \(0\;s\)    & \(0\;s\)                    \\
            \(\tau_{h}^{+}\) & -           & \(100\;ms\)                 \\
            \(\tau_{h}^{-}\) & -           & \(20\;ms\)                  \\
            \hline
        \end{tabular}
        \caption{\label{parameters-table}The parameters used for the LIF and IFB neurons, respectively.}
    \end{center}
\end{table}

Note that in the following all the differential equations are solved by means of the Euler
method and the selected time step is \(\Delta{t}=0.1\;ms\).

\section*{Question (a)}


\section*{Question (b)}

\section*{Question (c)}

\section*{Question (d)}

\end{document}