\documentclass[11pt]{article}
\usepackage{amsmath}
\usepackage{graphicx}
\usepackage{hyperref}
\usepackage{mathtools}
\usepackage[latin1]{inputenc}
% \usepackage{exercise}
\usepackage{cancel}
\usepackage[margin=1.25cm]{geometry}
\usepackage{float}
\usepackage{multirow}
% \usepackage{wrapfig}
% \usepackage{amssymb}
% \usepackage{subfigure}
\setlength\parindent{0pt}

\title{Computational Neuroscience Project - Group 04}
\author{Giuseppe Leo, Francesco Negri}
\date{A.Y. 2022-2023}

\begin{document}
\maketitle

\section*{Introduction}
The Leaky Integrate-and-Fire (LIF) and Integrated-and-Fire-or-Burst (IFB) models
are abstract neuronal models, aiming at reproducing the electrical physiological
behaviour of neurons.\\
Also, every time the membrane potential \(V_{m}\) overcomes the threshold voltage
\(V_{th}\) a spike is fired and the membrane potential goes back to \(V_{reset}\).\\
The LIF model is defined as followed:
\begin{align*}
    C_{m}\frac{dV_{m}}{dt} & =-I_{leak}+I_{stim}                           \\
                           & =-g_{leak}\bigl(V_{m}-E_{leak}\bigr)+I_{stim}
\end{align*}
On the other hand, the IFB neuron is modelled by these two differential equations:
\begin{align*}
    C_{m}\frac{dV_{m}}{dt}       & =-I_{leak}-I_{T}+I_{stim}                                                                                \\
                                 & =-g_{leak}\bigl(V_{m}-E_{leak}\bigr)-g_{T}h\bigl(V_{m}-E_{T}\bigr)\Theta\bigl(V_{m}-V_{h}\bigr)+I_{stim} \\
    \tau_{h}(V_{m})\frac{dh}{dt} & =-h+h_{\infty}(V_{m})
\end{align*}
where \(h_{\infty}(V_{m})=\Theta\bigl(V_{h}-V_{m}\bigr)\),
\(\tau_{h}(V_{m})=\tau_{h}^{-}\Theta\bigl(V_{m}-V_{h}\bigr)+\tau_{h}^{+}\Theta\bigl(V_{h}-V_{m}\bigr)\) and
\(\Theta\bigl(\,\cdot\,\bigr)\) is the Heaviside step function.\\
The two models are parametrized as shown in the table below:
\begin{table}[!h]
    \begin{center}
        \begin{tabular}{ |c||c|c|  }
            \hline
            \multicolumn{3}{|c|}{Models Parameters}                      \\
            \hline
            {}               & LIF neuron  & IFB neuron                  \\
            \hline
            \(S\)            & -           & \(3\cdot{10^{-4}}\;cm^{2}\) \\
            \(C_{m}\)        & \(100\;pF\) & \(2\;\mu{F}/cm^{2}\)        \\
            \(V_{0}\)        & \(-70\;mV\) & \(-58\;mV,\;\;-77\;mV\)     \\
            \(V_{reset}\)    & \(-70\;mV\) & \(-60\;mV\)                 \\
            \(V_{th}\)       & \(-50\;mV\) & \(-50\;mV\)                 \\
            \(V_{h}\)        & -           & \(-70\;mV\)                 \\
            \(V_{spike}\)    & \(20\;mV\)  & \(20\;mV\)                  \\
            \(g_{leak}\)     & \(10\;nS\)  & \(0.035\;mS/cm^{2}\)        \\
            \(E_{leak}\)     & \(-70\;mV\) & \(-75\;mV\)                 \\
            \(g_{T}\)        & -           & \(0.8\;mS/cm^{2}\)          \\
            \(E_{T}\)        & -           & \(120\;mV\)                 \\
            \(T_{ref}\)      & \(0\;s\)    & \(0\;s\)                    \\
            \(\tau_{h}^{+}\) & -           & \(100\;ms\)                 \\
            \(\tau_{h}^{-}\) & -           & \(20\;ms\)                  \\
            \hline
        \end{tabular}
        \caption{\label{parameters-table}The parameters used for the LIF and IFB neurons, respectively.}
    \end{center}
\end{table}

Note that in the following all the differential equations are solved by means of the Euler
method and the selected time step is \(\Delta{t}=0.1\;ms\).

\section*{Question (a)}
First row of Figure 1 shows the different trend 
of the membrane potential for the two models in
the without noise condition.
The applied current in both models is a step current with a 
constant value I0 summed with a step current I1 that starts at
time ton = 0.15 and ends at toff = 0.85. The total stimulation
time is 1 s. 
We choose values of current 
For the LIF we choose a starting current of 0 A and we applied
different stimulation currents of 190-250pA with a step of 2.5 pA. We choose a 
For IFB, as reported in the literature we choose a starting I0 
current of 230 pA and then we applied a step current I1 of.
Io+I1*(ton,toff)

We can see that in both cases for smaller current we have the
understhreshold condition: the voltage depolarizes but it 
doesn't reach the Vth so there aren't spikes.
In the suprathreshold case, the current is bigger and the neuron 
manages to reach the Vth so it fires. 
The firing frequency increases with the 
amplitude of the step current in both cases.



\section*{Question (b)}
Determine for the 2 models the corrispondent voltage 
threshold Vth: Analysing the gain function we calculate 
the bigger step current for having an underthreshold behaviour.
It is the threshold current between suprathreshold and 
underthreshold. Then we simulate our model using this threshold 
current as input and we find out that the maximum value of 
the membrane potential is almost -50 mV in both cases. 
This agree with our parameters, infact we set Vth to the 
same value.

\section*{Question (c)}
Figure 2 shows the gain functions of the two models. 
Also in all plot there is a zoom on the transition part that 
allows us to see the firing rate for values of step current 
near the threshold current.

The LIF neuron can be identified in the second class, in fact 
it's showed that it can't fire for at frequency less then 10 Hz.
The IFB neuron, instead is in class one, in fact it can fire at
almost all frequencies and the fire frequency is proportional 
to the amplitude of the current applied.
Also there is a difference in the behavior of the two neurons:
both models have a spiking behaviour, also the IFB model allows
to simulate the bursting activity, as we can see in the 3 Figure.
The difference in the model parameters between the two models is 
the starting membrane potential Vm(0) that is -58mV for tonic 
spiking and -77mV for the bursting activity.
Also we have to change the applied step current. In fact, 
in order to have bursts we need a bigger step current with 
respect to the tonic case.

\section*{Question (d)}
The added noise is a gaussian noise with mean 0 and standard 
deviation of 1/100 of the maximum value of the current.
Adding noise both models manage to spike at a lower 
firing frequency. -> Figure 2
IFB from 2 Hz to 1 Hz. Still class 1.
LIF from 15 Hz to 5 Hz. The model can fire at a lower frequency,
so it can have a behaviour more similar to a class 1 one.
An other difference with the without noise case is that here 
spikes are not fired at a constant frequency. This is due to the 
fluctuations in the amplitude of the current given by noise. 
-> Figure 1

\end{document}